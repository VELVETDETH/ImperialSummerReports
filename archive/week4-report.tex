
\section{Report of Week 4 (Aug 4th, 2015)}

This week I've made progress on all the 3 directions: SpMV in MaxJ, multi-pumping in MaxJ and multi-pumping in Ruby. Here're the details.

\subsection{SpMV in MaxJ}
According to the plan, I've finished the baseline version of CSRp SpMV on time. You could find out the definition of CSRp in the archived week 3 report. This version will generate correct result, but it must be improved in many aspects:
\begin{enumerate}
\item 
\textbf{Get runtime correct}: Currently, if we add level of parallelization, the run time will increase, not as what we expected. If the code is correct(maybe not), then something wrong might happen during synthesis and implementation. Check the \textbf{clock frequency} of the \textbf{built kernel}. This will tell whether there's problem during the implementation: if the clock frequency also not follow our expectation.

It might be the case that the current using matrix will never benefit from the level of parallelization, so we need to generate matrix based on our current design specification, and discover their performance.

\textbf{Todo} First check out the frequency of each design with different parallelization level. Also try to organize the code better.Then use randomized benchmark, not only the value, but also the structure to test run time result. 

\item
\textbf{Improved CSRp format}: The CSRp format will perform really bad if the padded zeros occupy most of the computations, which means, we have lots of redundant zeros. Why we have those zeros? As we cannot calculate all the rows simultaneously, only part of the rows will be calculate at the same time. If we call these rows a \textbf{block}, then we will calculate the whole matrix block by block. Now, we do not specifically add \textbf{splitters} between those blocks. We would know which block we are calculating only by forcing all the blocks has equal width. The way we force this happen is by padding zeros. The number of padding zeros could be enormous if we have a very wide block and most of the blocks are quite narrow. The \textbf{redundancy} metrics could be calculated by using:

\[ R_{CSRp}(N) = 
	\max_{0 \leq i < N}(block_i.width)\times N \times p - 
	\sum_{i}^{Np} row_i.width 
\]

where N is the number of blocks, and p is the level of parallelization. If we have N and p assigned, then the whole number of rows should be Np. The \textbf{width} is the number of non-zeros for each row or block. Especially for blocks, their width is the maximal width among all rows in that block.

Obviously, we could improve this. If we assign more information for each clock cycle, like "whether this cycle is in the end of one block", which is the \textbf{splitter} that we have mentioned above, then we do not need to do padding on all the blocks to make sure they have same width.

This one could be achieved by adding a new input stream of 0, 1 boolean values. We could verify the improvement in speed and space increasement.

\textbf{Todo}: Finish the first part of SpMV work first. And build this design later. Discover their performance differences. And I think this is one should be the standard design of \textbf{baseline}. Previous one has really bad performance.

\item
\textbf{With multi-pumping}: I've already finished a multi-pumping MaxJ version, which has a simple multiplier inside. And we all know that SpMV is all about add and multiply(basically), so we could use the same framework to optimize this SpMV design.

\textbf{Todo}: Last one to do, I think I could do this in the next week.
\end{enumerate}

That's all about this week's SpMV work, I'll attach a full chart about this design's performance in next week's report, as I need to verify some current data this week.

\subsubsection{Multi-pumping in MaxJ(Updated on Aug 25th 2015)}

In this part I'll just illustrate some result I've gathered in my Multi-pumping MaxJ version's experiment. 

First of all, we have these versions to implement the simple multiplication functionality:
\begin{enumerate}
\item \textbf{UnrollLoop}: This one will use 1x clock rate for multiplier, but will use 2 of them.
\item \textbf{ResourceShare}: This one will use one single multiplier to do 2 multiplications sequentially, at 1x clock rate.
\item \textbf{Multi-pumping}: This one will use 1 multiplier, running at 2x clock rate. And other parts running at 1x clock rate. 
\end{enumerate}

What we expect about these versions are:
\begin{enumerate}
\item Multi-pumped version will save number of DSPs compared to the origin version, which will use 2 multiplier to finish 2 multiplication in one 1x clock rate cycle. 
\item Multi-pumped version will be 2x faster than the resource sharing version, which will not use more multiplier, just finish the 2 multiplication one by one. 
\end{enumerate}

And the result looks the same:
\begin{enumerate}
\item \textbf{Speed}: UnrollLoop version has the fastest frequency(178MHz), and multi-pumping has similiar result(171MHz). ResourceShare version could just achieve have of the frequency(92.1MHz).
\item \textbf{Resource Usage}: For the DSPs, it's obvious that UnrollLoop will use 2 and the others two version will use one. And it should be true that Multi-pumping will have more logics usage(14479) that ResourceShare(13923) as it will have more control logic.
\end{enumerate}

